\documentclass[10pt]{article}
\usepackage{geometry}
\usepackage{csquotes}
\geometry{
    a4paper,
    total={170mm,257mm},
    left=20mm,
    top=15mm,
    }

    
\title{Statement of purpose for application to Masters program in Quantum Science and Engineering} % Sets article title
\author{Sourabh Choudhary} % Sets authors name
\date{\today} % Sets date for date compiled

\begin{document}
\pagenumbering{gobble}
\maketitle

I aim to have a more educated understanding of the impact that quantum technology will
have on the industry, how this technology will develop. 
With this statement, I will convey my interest in quantum technology and also 
demonstrate the skills and knowledge I have gained from my academic and extracurricular 
experience that will support my application to the program.

\vspace{5mm}

Within the domain of hardware development for quantum computing and computing in general, I am 
interested in nano-scale standalone and hybrid photonic systems complemented by advanced quantum 
architecture. In the corresponding domain of software development, I am interested in quantum 
computational science for development the of efficient algorithms. The active research in these 
fields conducted at two pillars of Quantum Science Engineering at EPFL is what motivates me to 
apply for the masters program.

\vspace{5mm}

I have studied introductory quantum optics and atomic physics in my undergraduate degree. These 
courses have helped me gain an understanding of the quantised light-matter interactions. They have
also given an insight into how these principles can be applied to make novel devices in quantum 
technology. Extracurricular activities related to these topics led me to gain some programming 
experience for quantum computers. I used IBM's Quantum Cloud Computing service to test a simplified 
version of Shors algorithm. 

\vspace{5mm}

To improve my interpersonal and leadership skills I founded the \enquote*{RoboClyde Society} at
our University. This was the first team from Scotland to participate in the 
European Rover Challenge, where students from universities around the world come together to build 
a demonstration mars rover. Our team had members from the Science, Engineering, and Business faculties
at the university. Leading such an interdisciplinary organisation has given me an insight into the 
challenges that come when working with people from different academic backgrounds. Together we 
won the $3^{rd}$ place in the remote edition of the European Rover Challenge in 2021.

\vspace{5mm}

My final year project \enquote*{Photon statistics of micro to nano scale lasers at threshold} gave me 
experience with quantum optical experimental methods. I first conducted  experiments on the emission 
from a microlaser, and collected data for analysis. Using python I analysed the data to characterise
the emission and study the noise properties of these lasers near threshold. Throughout the project I 
also learned about the potential application of these lasers. I am specifically interested in their 
application in photonic 
classical and quantum computing.

\vspace{5mm}

I would also like to briefly address the Experimental Physics II module that I \enquote*{failed}
in Year 3 of my course. This was during COVID, when I experienced some financial difficulties 
and had to do extra part-time work which made it impossible for me to attend the lab sessions. 
That time was difficult, managing part-time work and university responsibilites alongwith my 
personal project, \enquote*{RoboClyde} all in the midst of lockdowns was stressfull. Regardless, I 
did the best I could in everything I was able to work on and continue to do so.

\vspace{5mm}

It is clear that there is a lot of potential for new companies and services to emerge from quantum
tech. For example software for modelling quantum systems, bridging classical and quantum computers,
fabrication of components for quantum tech, etc. Since this field is still in its infancy in the 
industrial scene, having a strong academic background is essential to keep up with the changes. This 
is why am pursuing the masters program in Quantum Science and Engineering at EPFL.

\end{document}